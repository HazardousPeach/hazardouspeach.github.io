\documentclass[11pt,a4paper,sans]{moderncv}

\moderncvstyle{casual}
\moderncvcolor{green}

\usepackage[scale=0.75]{geometry}

\firstname{Alex}
\familyname{Sanchez-Stern}

\title{Curriculum Vitae}
\address{3418 NE 65th St \#1}{Seattle, WA 98115}
\phone{(206) 902 6299}
\email{alex.sanchezstern@gmail.com}
\homepage{alexsanchezstern.com}
\photo[70pt][0.4pt]{images/me4}

\begin{document}

\makecvtitle

% Publications
\section{Publications}

\cventry{December 2023}{Passport: Improving Automated Formal Verification Using Identifiers}{TOPLAS 2023}{}{}{}
\cventry{May 2023}{Proofster: Automated Formal Verification}{ICSE 2023 Demo Track}{}{}{}
\cventry{December 2022}{Data Driven Lemma Synthesis for Interactive Proofs}{OOPSLA2022}{}{}{}
\cventry{June 2021}{Scooter \& Sidecar: a domain-specific approach to writing secure migrations}{PLDI 2021}{}{}{}
\cventry{June 2020}{Generating Correctness Proofs with Neural Networks}{MAPL 2020}{}{}{}
\cventry{January 2020}{REPLica: REPL Instrumentation for Coq Analysis}{CPP 2020}{}{}{}
\cventry{June 2018}{Finding Root Causes of Floating Point Error}{PLDI 2018}{}{}{}
\cventry{July 2016}{Towards a Standard Benchmark Format and Suite for Floating-Point Analysis}{NSV 2016}{}{}{}
\cventry{June 2015}{Automatically Improving Accuracy for Floating Point Expressions}{PLDI 2015}{Distinguished Paper Award}{}{}

\section{Awards}

\cvitem{2015}{Marygates Research Scholarship}
\cvitem{2015}{Distinguished Paper -- PLDI 2015}

\section{Service}
\cvitem{2023}{OOPSLA ERC}
\cvitem{2021}{AIPLANS Committee}
\cvitem{2020-2021}{ACM Mentorship Program Mentor}
\cvitem{2019-2020}{ICFP Artifact Evaluation Committee}
\cvitem{2018-2019}{POPL Student Volunteer Captain}

%----------------------------------------------------------------------------------------
%      EDUCATION
%----------------------------------------------------------------------------------------

\section{Education}

\cventry{2018--2021}{Doctor of Philosophy, Computer
  Science}{University of California, San Diego}{}{}{}
\cventry{2016--2018}{Candidate of Philosophy, Computer
  Science}{University of California, San Diego}{}{}{}
\cventry{2015--2016}{Masters of Science, Computer Science}{The
  University of Washington}{}{}{Honors}
\cventry{2012--2015}{Bachelors of Science, Computer Science}{The
  University of Washington}{}{}{Honors}

\section{PhD Thesis}
\cvitem{Title}{\emph{Hybrid-Neural Synthesis of Machine-Checkable Software Correctness Proofs}}
\cvitem{Supervisor}{Professor Sorin Lerner}

\cvitem{Description}{The correctness of large software artifacts has
  important impact on many aspects of the modern
  world. Machine-checkable software correctness proofs provide a
  guarantee that a piece of software adheres to some logical
  specification, however producing such proofs is labor-intensive,
  taking in some cases 23 person-years of highly skilled labor to
  prove properties of 10,000 line programs. This thesis work uses a
  hybrid-approach of machine learning and proof assistant search
  procedures to produce proofs of correctness for a large variety of
  software automatically or semi-automatically.}

\section{Masters Thesis}
\cvitem{Title}{\emph{Dynamic Analysis of Floating Point Errors with Herbgrind}}
\cvitem{Supervisor}{Professor Zachary Tatlock}
\cvitem{Description}{Numerical computation using floating point
  numbers is notoriously difficult to reason about, even in idealized
  environments. This thesis presents the development of a tool which
  can analyze the runtime behavior of programs written in a variety of
  environments and languages, and extract inaccurate floating point
  computation for improvement.}

\section{Bachelors Thesis}

\cvitem{Title}{\emph{Algebraic Simplification for the Herbie Project}}
\cvitem{Supervisor}{Professor Zachary Tatlock}
\cvitem{Description}{The ability to simplify arbitrary mathematical
  expressions is extremely useful in many applications, including the
  Herbie numerical synthesis tool, but is exponential in general. This
  thesis presents a set of data structures and heuristics that allow
  thousands of expressions to be simplified every second.}

%----------------------------------------------------------------------------------------
%	WORK EXPERIENCE
%----------------------------------------------------------------------------------------

\section{Experience}

\subsection{Vocational}
\cventry{Sepember 2021--Present}{Postdoctoral
  Researcher}{\textsc{University of Massachusetts, Amherst}}{Amherst,
  MA}{}{Worked on an extension to the TacTok proof synthesis tool with
  co-authors at UMass Amherst and University of Illinios, Urbana
  Champagne. Also advised a new PhD student in her studies, worked on
  a masters students project on localization of errors in flakey
  tests, and extended thesis work with reinforcement learning
  concepts.
\newline{}\newline{}
\begin{itemize}
\item Co-PI on a DARPA-PEARLS proposal.
\item Submitted a paper to PLDI in my first three months, on inferring helper
  lemmas for Coq proofs.
\item Now have 3 papers published during the postdoc with 3 more in
  the pipeline.
\end{itemize}}

\cventry{September 2016--June 2021}{Research
  Assistant}{\textsc{University of California, San Diego}}{San
  Diego}{}{Continued work begun at the University of Washington on the
  Herbgrind project for automatically diagnosing the causes of
  floating-point error in large numerical software, and began work on
  neural synthesis of machine-checkable proofs of program correctness.
\newline{}\newline{}
Detailed achievements:
\begin{itemize}
\item Worked with collaborators at UCSD to produce Proverbot9001, a
  tool for neural proof synthesis.
  \begin{itemize}
  \item Implemented in Python using PyTorch and Rust
  \item Can find proofs for almost a quarter of all theorem statements
    in CompCert (a verified C compiler).
  \item Published and presented as ``Generating Correctness Proofs with Neural Networks'' at MAPL 2020
  \item Pre-print of the paper available at
    \url{http://proverbot9001.ucsd.edu/papers/proverbot9001.pdf}
  \item Talk is available as part of MAPL proceedings at
    \url{https://youtu.be/rwBbYhOAnPo?t=11540}
  \end{itemize}
\item Worked with Collaborators in the Systems \& Security groups to
  produce Scooter, a tool to make data migrations safer.
\item Worked with collaborators at the UW as well as Sorin Lerner at
  UCSD to complete work on the Herbgrind tool and paper.
  \begin{itemize}
  \item Implemented in 20,000 lines of code (C, python, and bash
    scripts).
  \item Analyses programs up to 50,000 lines of code.
  \item Published and presented ``Finding Root Causes of Floating
    Point Error'' at PLDI 2018
  \item Pre-print of the paper available at
    \url{http://herbgrind.ucsd.edu/herbgrind-pldi18.pdf}
  \item Talk slides available at
    \url{http://herbgrind.ucsd.edu/pldi18-talk/}
  \item Talk video available at
    \url{https://www.youtube.com/watch?time_continue=1&v=bFL6PaPrz8Y}
  \end{itemize}
\item Continuing maintenance of the Herbie project with collaborators
  at the UW.
\end{itemize}}

\cventry{December 2013--September 2016}{Research Assistant}{\textsc{University of
    Washington}}{Seattle}{}{Worked with another research assistant to
  develop the Herbie system for automatically improving the accuracy of
  floating point code
\newline{}\newline{}
Detailed achievements:
\begin{itemize}
\item Worked with Pavel Panchekha and Zachary Tatlock in developing
  the high level design of the system over the course of two years.
\item Worked closely with Pavel Panchekha to write the implementation
  of the system, including specifically:
  \begin{itemize}
  \item Independently developing the algebraic simplification system
  \item Writing the top level code which controls the various subsystems
  \item Developed the experimental loop variant of Herbie to continue
    the work described in the paper.
  \end{itemize}
\item Authored a paper on our work together with Pavel Panchekha,
  Zachary Tatlock, and James Wilcox.
  \begin{itemize}
    \item Our paper was published at the Programming Languages Design
      and Implementation 2015 conference.
    \item Paper and talk available at \url{http://herbie.uwplse.org/pldi15.html}
  \end{itemize}
\item Authored a second paper with Pavel Panchekha, Zachary Tatlock,
  Chen Qiu, and international collaborators Nasrine Damouche and
  Matthieu Martel on a new format and benchmark suite for cross-tool
  floating point benchmarks.
\item Began work on a third project, Herbgrind, which I continued at UCSD
\end{itemize}}

\cventry{June 2013--September 2013}{College Tech}{\textsc{Seattle
    Schools District}}{Seattle}{}{Maintained existing educational and
  teacher machines, and set up and installed new machines, at a
  variety of schools in the Seattle Schools District.}

\cventry{September 2011--January 2013}{Assistant Operations
  Engineer}{\textsc{Casa Latina}}{Seattle}{}{Wrote tests and data aggregation and
  display code for the Machete job registration system, under James
  Carter.}

\cventry{July 2011--September 2011}{Intern}{\textsc{Bensussen Deutsch
    \& Associates, Inc}}{Woodinville}{}{Performed market research,
  handled product returns, and managed product testing.}


%% %----------------------------------------------------------------------------------------
%% %	COMPUTER SKILLS SECTION
%% %----------------------------------------------------------------------------------------

%% \section{Technical Skills}

%% \cvitem{Basic}{\textsc{java}, \textsc{haskell}, \textsc{coq}, \textsc{rust}}
%% \cvitem{Intermediate (authored projects with at least hundreds of lines of code)}{\textsc{HTML/Javascript/CSS}, \textsc{c\#}, \textsc{c++}, \LaTeX, Linux}
%% \cvitem{Advanced (authored projects with at least thousands of lines of code)}{\textsc{racket}, \textsc{c}, \textsc{python}}

\end{document}
